\documentclass{report}
\begin{document}
\section*{Anuncios}


\section*{El cielo como fuente de ritos}
En las sociedades humanas primitivas es com\'un la existencia de
divinidades que habitan el cielo.
Dan vida, ponen el mundo en movimiento, velan por un orden o lo
instauran inicialmente. 

La contemplacion misma del cielo produce esta experiencia religiosa y
de conecci\'on con el Universo. Es casi una revelaci\'on. Experiencia
sublime. \cite{Eliade}

El cielo es casi que por excelencia el lugar que es totalmente
diferente a lo que tenemos en nuestro nivel. Casi
inalcanzable. Solamente este atributo "alt\'isimo", "inalcanzable" se
convierte en una caracter\'istica de la divinidad. As\'i mismo acceder
a ese espacio superior solamente es posible a trav\'es de ritos
especiales de ascenci\'on \cite{Eliade}.

El cielo, desde la feonomeonolog\'ia m\'as pura de la experiencia
inmediata, ya es "elevado, inmutable y poderoso" \cite{Eliade}.

"Los Sioux expresean la fuerza m\'agico religiosa con el t\'ermino
\emph{wakan}, pr\'oximo fon\'eticamente de  \emph{wak\'an} que en la
lenguna dakota significa arriba, encima" \cite{Eliade} 

"La divinidad suprea de loa maor\'ies se nombra Iho", palabra que
tambi\'en tiene el sentido de elevado, arriba.

Todo lo que pueda suceder en esos espacios celestes en las partes
superiores de la atm\'osfera (periodicidad del movimiento de los
astros, las nuves, el arcoiris) se pueden integrar inmediatamente  ese
repertorio revelaci\'on de lo sagrado (hierofan\'ia).  


\section*{Dioses del Cielo en diferentes culturas}


\begin{itemize}
\item Suroriente de Autralia. "Baiamae es la divinidad suprema y
  habiata el cielo junto a un gran camino de agua (la v\'ia
  l\'actea) donde recibe las almas de los inocentes. Esta sentado
  sobre un trono de cristal. El Sol y la Luna son sus hijos,
  observadores de la tierra."\cite{Eliade}.

"Otras tribus conocen a Darmulun, un nombre que solamente es
  comunicado a iniciados. Las mujeres lo conocen como \emph{padre} y los
  ni\~nos como \emph{Senor}" \cite{Eliade}.

\item Africa. Dioses perif\'ericos. Muy buenos o muy alejados de loq
  euasa en el nivel humano.

\item Amazonas. Ax-Piconda. La anaconda celeste. Ordena el territorio,
  conecta la parte de arriba con la parte de abajo. Nuestro nivel es
  un nivel intermedio. 

\item Grecia. Uranos. P\'agina 90. Zeus (D\'ias en griego moderno)
  P\'agina 93.
\end{itemize}

Es bastante com\'un que estas deidades celestes est\'an alejadas de la
vida diar\'ia. Al nivel terrestre se reemplazan por representantes o
por formas diferentes de  culto de esp\'iritus o seres m\'as
terrestres, en muchos otros casos son entidades solares o lunares. 

Ejemplo de Am\'erica del Norte de esta sustituci\'on (p\'agina 71).

Otro objeto claro de esta sustituci\'on es el trueno.  El rayo es el
arma de dios en varias mitolog\'ias. 

Y a este nivel ya no hay sino un paso mas a la filosofia. Contemplar
el cielo, reconocer la divinidad y convertirse en una revelacion sobre
la precariedad el hombre, de la trascendencia de la divinidad, pero al
mismo tiepo de una revelaci\'on de lo sagrado del
conocimiento. Aquella persona que "ve y comprende". Que sabe.

Esta representaci\'on baja hasta niveles de humanos que son hijos de
deidades o del Sol \cite{Eliade} (Pag 78-80). De esto hablaremos en
detalle cuando tratemos el tema de Astronom\'ia y Poder.


{\bf Un desv\'io. Que significado tiene tomar estas clases ahora en la
Universidad comparado con alguien que no tiene acceso a este
conocimiento?
Esto se puede presentar como un ritual inici\'atico donde se devela el
verdadero nombre de la divinidad (ley natural? ley de origen?), hay
una ceremonia que asegura la salvaci\'on del iniciado (exaltaci\'on de
una salida de un estado de ignorancia, asegura un "futuro" material) y se hacen
revelaciones sobre el principio y origen del Universo. }(p\'agina 74
de Eliade). 


\section*{Simbolog\'ias de los di\'oscuros}

Los di\'oscuros son hijos de Zeus. Son gemelos. Pero los mencionamos
para hablar de un arquetipo conocido. Son hijos del dios celeste con
una mortal y tienen siempre una actitud benefactora en la
tierra. (Eliade P\'agina 111)
 

\section*{Puntos importantes de Eliade}
\begin{itemize}
\item La b\'oveda celeste es en s\'i misma un elemento rico en valores
  m\'iticos y religiosos,
\item Los habitantes de la b\'oveda celeste se asocian inmediatamente
  con seres supremos. Logica
  propia. Buenos. Creadores. Eternos. Fundadores de instituciones. 
\item Estos seres suelen ser centrales en mitos fundacionales pero en
  el culto diario se reemplazan por entidades m\'as cercananas a la
  vida diar\'ia, en muchos casos una deidad Solar, o el mismo Sol.
\item Aun cuando la vida celeste no est\'e dominada por los dioses o
  ritos celestes se guarda una concepci\'on de ritos de ascencion o de
  conexion con lo celeste (piramides, montanas sagradas)
\end{itemize}

\section*{Cultos Solares}

A pesar de su apariencia universal, el culto solar solamente fue
preponderante en Egipto, Asia y Europa, y en America en M\'exico y
Per\'u. 

\section*{Ejemplos de cultos solares}

\begin{itemize}
\item Egipto
\item Mesopotamia
\item India
\end{itemize}






\section*{Solsticios y equinoccios}

Equinoccio es el momento del a\~no en el que el Sol est\'a situado en
el plano del Ecuador terrestre.

Solsticios son los momentos en los que el Sol alcanza su mayor o menor
altura aparente en el cielo.

\begin{itemize}
\item Equinoccio de oto\~no. (21 de septiembre)
\item Solsticio de invierno. (21 de diciembre)
\item Equinoccio de primavera. (21 de marzo)
\item Solsticio de verano. (21 de junio)
\end{itemize}

En la Plaza de Bolivar, los d\'ias de los equinoccios el sol sale
entre Monserrate (Tensaca) y Guadalupe (Guafa) [Segun Mariana
  Escribano]. En el solsticio de junio sale por Monserrate y en el
solsticio de diciembre por Guadalupe. 

\section*{El ciclo solar}
En la clase que viene hablaremos de simbologia lunar. A gran
diferencia del sol, la luna parece tener cambios dr\'asticos de
desaparici\'on y aparicion.

\begin{itemize}
\item Heliosimolog\'ia. Oscilaci\'on de cinco minutos. Heliosismolog\'ia.
\item Ciclo solar. Ciclo de 11 a\~nos. Estamos en el Ciclo solar
  24. Ahora estamos saliendo de un
  m\'aximo. 
%https://en.wikipedia.org/wiki/Solar_cycle_24#/media/File:Solar_cycle_24_sunspot_number_progression_and_prediction.gif 

\end{itemize}

\end{document}

Referencias:

- Aproximaciones al observatorio solar de Bacatá-Bogotá-Colombia. Julio Bonilla. 

Para tener en cuenta mas tarde.
- Cosmologia y alquimia babilonicas.
