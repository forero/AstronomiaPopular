\documentclass[letterpaper,10pt,onecolumn]{article}
\usepackage[spanish]{babel}
\usepackage[latin1]{inputenc}
\usepackage[pdftex]{color,graphicx}
\usepackage{hyperref}
\setlength{\oddsidemargin}{0cm}
\setlength{\textwidth}{490pt}
\setlength{\topmargin}{-40pt}
\addtolength{\hoffset}{-0.3cm}
\addtolength{\textheight}{4cm}

\begin{document}
\begin{center}

\includegraphics[width=490pt]{header.png}\\[0.5cm]

\textsc{\LARGE Astronom\'ia Popular}\\[0.1cm]

\large Jaime E. Forero Romero\\[0.5cm]

\end{center}

\large \noindent\textsc{Nombre del curso:} Astronom\'ia Popular %Aqui
                                %nombre del curso 
 
\noindent\textsc{C\'odigo del curso:} FISI-1916B. CBU B.

                                %codigo del curso 

\noindent\textsc{Unidad acad\'emica:} Departamento de F\'isica 

\noindent\textsc{Periodo acad\'emico:} 201620 %Aqui el periodo,
                                %p.ej. 201510 

\noindent\textsc{Horario:} Mi y Vi, 11:30 a 12:50 %Aqui el horario, p.ej. Ma y Ju, 10:00 a
                           %11:20 

\noindent\rule{\textwidth}{1pt}\\[-0.3cm]

\normalsize \noindent\textsc{Nombre profesor(a) principal:} Jaime
E. Forero-Romero %Aqui nombre del profesor principal 

\noindent\textsc{Correo electr\'onico:}
\href{mailto:je.forero@uniandes.edu.co}{\nolinkurl{je.forero@uniandes.edu.co}}
%Cambie address por su direccion de correo uniandes 

\noindent\textsc{Horario y lugar de atenci\'on:} Lu 14:00 a 15:00, Oficina Ip208 %Aqui su
                                %horario y lugar de atencion,
                                %p.ej. Vi, 15:00 a 17:00, Oficina
                                %Ip102 
\\[-0.1cm]


\noindent\rule{\textwidth}{1pt}\\[-0.1cm]

\newcounter{mysection}
\addtocounter{mysection}{1}

\noindent\textbf{\large \Roman{mysection} \quad
  Introducci\'on}\\[-0.2cm] 

%Este espacio es para hacer una introduccion al curso, evidenciando la propuesta metodologica. Debe ser clara y precisa.

\noindent\normalsize 
La astronom\'ia est\'a presente en la informaci\'on que
recibimos a diario en el Siglo XXI. Posible vida en otros planetas, el
origen del  Universo, apocalipsis astron\'omicos, ciencia ficci\'on,
celebraciones rituales en el calendario y la exploraci\'on espacial
son algunos de los temas que parecen familiares a la mayor\'ia pero que
en realidad no los conocemos en toda la profundidad que merecen. El
objetivo principal del curso es tratar aspectos de la cultura popular
relacionados con la astronom\'ia y presentarlos a trav\'es del
conocimiento que se ha constru\'ido en cosmolog\'ia, astronom\'ia
gal\'actica, astrof\'isica planetaria, astrof\'isica estelar,
astronom\'ia de posici\'on, coheter\'ia, ingenier\'ia aeroespacial y
astrobiolog\'ia. \\[0.1cm]  

\stepcounter{mysection}
\noindent\textbf{\large \Roman{mysection} \quad Objetivos}\\[-0.2cm]

%En este espacio se debe precisar el ente visor del curso y el proposito ideal al finalizar el curso.

\noindent\normalsize Los objetivos principales del curso son:

\begin{itemize}
\item Reconocer el lugar del conocimiento astron\'omico en diferentes
  aspectos de la cultura popular. \\[-0.6cm] 

\item Examinar los v\'inculos hist\'oricos de la astronom\'ia con
  otras \'areas del conocimiento.  \\[-0.6cm] 

\item Reconocer diferentes maneras de construir y validar conocimiento
  en las ciencias naturales, en particular en astronom\'ia.\\[-0.2cm]
\end{itemize}

\stepcounter{mysection}
\noindent\textbf{\large \Roman{mysection} \quad Competencias a
  desarrollar}\\[-0.2cm] 

%En este espacio se describen las habilidades que el estudiante
%desarrollara en el transcurso del curso. 

\noindent\normalsize Al finalizar el curso, se espera que el
estudiante est\'e en capacidad de: 

\begin{itemize}
\item Analizar contenidos de diferentes fuentes en la cultura popular
  para encontrar su relaci\'on con el conocimiento astron\'omico.\\[-0.6cm]
\item Leer textos no especializados con contenido astron\'omico para
  reconocer el posible uso correcto o incorrecto de conceptos cient\'ificos.
\\[-0.6cm]
\item Escribir un texto sobre alg\'un aspecto de la cultura popular
  influenciado por la astronom\'ia para evidenciar la relaci\'on del
  conocimiento cient\'ifico con las creencias/valores del p\'ublico no
  especializado.\\[-0.2cm]  
\end{itemize}

\stepcounter{mysection}
\noindent\textbf{\large \Roman{mysection} \quad Contenido por
  semanas}\\[-0.2cm] 

%Se expone de forma ordenada toda la tematica a tratar del curso. Debe
%planearse para 15 semanas. 

\noindent\normalsize \textbf{\textsc{Semana 1.}} \textit{El Comienzo. Las
diferentes historias de origen en los \'ultimos dos mil a\~nos
en diferentes civilizaciones de Asia, Africa y Am\'erica. La teor\'ia
del Big Bang.} Las historias de or\'igen son parte integral de una
civilizaci\'on. Empezaremos esta clase pregunt\'andole a algunas
personas c\'omo creen que empez\'o todo. A partir de esto veremos que
nuestras nociones impl\'icitas sobre la creaci\'on generan un
v\'inculo entre nosotros. Repasaremos c\'omo diferentes visiones de la
creaci\'on han dado forma a civilizaciones en diferentes
\'epocas y lugares. Terminaremos por presentar un resumen sobre la
historia de creaci\'on actual aceptada por la ciencia: el Big Bang.\\[-0.3cm]   


\noindent\textbf{\textsc{Semana 2.}} \textit{Espacio y tiempo. Modelos de
espacio y tiempo desde Grecia antigua hasta Einstein.} Un segundo
elemento que une a una sociedad es el concepto de espacio y de
tiempo. Mostraremos c\'omo la forma en la que se mide y la forma en la que se crea una
estructura mental de estos dos conceptos define la
posici\'on ante la vida de sociedades completas. La estructura del
espacio (finito, infinito, lineal, curvo), el tiempo (simult\'aneo,
antes, despu\'es), la sensaci\'on de espacialidad del tiempo (i.e. el
futuro est\'a \emph{adelante nuestro} o \emph{detr\'as   nuestro}) son
temas que trataremos para pasar a mostrar expl\'icitamente las
construcciones mentales que rigen nuestra manera actual de sentir y
medir el espacio/tiempo, en la ciencia y en la vida diaria. (Ver nota
de prensa en la Bibliograf\'ia Complementaria).\\[-0.3cm]  

\noindent\textbf{\textsc{Semana 3.}} \textit{Ritmos de la vida diaria. Ciclos
lunares y solares que dictan ritmos biol\'ogicos y
dan estructura a sociedades.} Nuestros ritmos compartidos de
actividad/descanso a lo largo del d\'ia y del a\~no son definidos puramente por ciclos del
movimiento de la Tierra al rededor del Sol. Revisaremos las diferentes
posibles definiciones y mediciones de estos ciclos y su influencia
sobre los ritmos biol\'ogicos (i.e. producci\'on de hormonas
estimulados por la luz solar). Igualmente revisaremos los ciclos
humanos que tienen una periodicidad lunar en agricultura y en ritmos
biol\'ogicos (i.e. el ciclo menstrual de las mujeres). \\[-0.3cm]  

\noindent\textbf{\textsc{Semana 4.}} \textit{Ritmos de la vida
religiosa. Bases astron\'omicas presentes en la celebraci\'on de ritos
religiosos.} Mostraremos c\'omo las celebraciones de religiones
como el cristianismo  fueron tomadas de ritos paganos que celebraban las
transiciones entre estaciones en el hemisferio norte.  Hablaremos de
c\'omo la mitolog\'ia de un enviado de dios o un ser 
superior que muere y resucita un tercer d\'ia para subir al cielo ha
sido recurrente en la historia de la humanidad con los mitos de
Tammuz, Osiris, Adonis y Cristo. Una interpretaci\'on es que estos
mitos se  refieren a un culto simb\'olico al Sol y sus ciclos.  \\[-0.3cm] 

\noindent\textbf{\textsc{Semana 5.}} \textit{Signos. Lo que la ciencia
contempor\'anea tiene que decir sobre la astrolog\'ia.} Los
hor\'oscopos y cartas celestes para adivinar el futuro tal vez sean
uno de los fen\'omenos globales m\'as populares. Luego de
mostrar c\'omo las bases conceptuales de la astrolog\'ia (un destino
determinado por los astros) no tiene ning\'un fundamento cient\'ifico,
pasaremos a mostrar formas reales en los que eventos astron\'omicos pueden
influir sobre la poblaci\'on humana (i.e ciclos de actividad solar y
meteoritos).\\[-0.3cm]  

\noindent\textbf{\textsc{Semana 6.}}\textit{El Espacio visto por los
pol\'iticos. La carrera espacial y su lugar en luchas de poder global
durante el Siglo XX y XXI.} La carrera espacial populariz\'o la idea
del espacio en el gran p\'ublico. Impulsadas por la propaganda
Americana y Sovi\'etica, las ideas sobre la conquista espacial
llenaron la televisi\'on, el cine, la radio, los peri\'odicos y los
c\'omics. Hoy en d\'ia la conquista espacial sigue siendo un signo
claro de poder estrat\'egico y militar como muestran los avances de
China e India. En esta semana nos dedicaremos a mostrar como los
avances cient\'ificos y tecnol\'ogicos se usan como un instrumento
pol\'itico. \\[-0.3cm]  

\noindent\textbf{\textsc{Semana 7.}} {\textit El Universo visto por los
periodistas. Los errores y aciertos de los periodistas en
cubrimientos de noticias sobre astronom\'ia.} El periodismo
no siempre es fiel a la realidad de lo que pas\'a en las
investigaciones cient\'ificas y en varios casos simplemente se trata
de amarillismo para aumentar el n\'umero de lectores. Vamos a revisar
ambos casos, empezando por intentos de periodismo cient\'ifico que no
logran su cometido, incluyendo el uso y abuso de visualizaciones
art\'isticas. En la siguiente parte vamos a revisar y comentar noticias sobre
OVNIS, marcianos, abducciones, historias de apocalipsis y errores
astron\'omicos (i.e. "Marte se ver\'a m\'as grande que la
Luna!") recurrrentes en los medios masivos de comunicaci\'on.)\\[-0.3cm]    

\noindent\textbf{\textsc{Semana 8.}} \textit{?`Existen OVNIS y
marcianos?. Astrobiolog\'ia b\'asica y la b\'usqueda cient\'ifica de
vida  extraterrestre.} Empezaremos por revisar noticias en medios
masivos de comunicaci\'on sobre avistamientos de OVNIS y encuentros
con seres extraterrestres. Esto nos servir\'a de base para saber qu\'e
limitaciones tienen esas fuentes de informaci\'on y pasar a
mostrar lo que sabemos cient\'ificamente del tema. Con esto haremos
una transici\'on para presentar el \'area de estudio de la
Astrobiolog\'ia y el t\'ipo de preguntas que se hacen actualmente en
diferentes centros de investigaci\'on. \\[-0.3cm]   

\noindent\textbf{\textsc{Semana 9.}} \textit{Nuevos mundos. Exoplanetas y la
b\'usqueda de nuevos mundos habitables.} La b\'usqueda de nuevos
mundos ha estado presente como una posibilidad real desde la carta de
Kepler a Galileo: \textit{"There will certainly be no lack of human
  pioneers when we   have mastered the art of flight....Let us create
  vessels and sails   adjusted to the heavenly ether, and there will
  be plenty of people   unafraid of the empty wastes. In the meantime we shall prepare, for
  the brave sky-travelers, maps of the celestial
  bodies."}, pero fue solamente hasta finales del siglo XX con el
descubrimiento observacional de otros planetas que esto dejo de ser
una especulaci\'on y la b\'usqueda de mundos habitables se convirti\'o
en un \'area activa de investigaci\'on. Esto nos servir\'a como
introducci\'on al tema de la siguiente semana sobre astronautas y
posibles viaje de humanos a otros planetas.\\[-0.3cm]    
 
\noindent\textbf{\textsc{Semana 10.}} \textit{?`Qu\'e se necesita para ser un
astronauta?. Humanos (y animales) en el espacio. Historias  de
cosmonautas y astronautas. ?`Llegaremos a Marte en el 2025?.} Dedicaremos una primera parte de la semana a revisar la historia de la carrera espacial por enviar humanos al espacio. Desde
los intents con Laika y otros animales hasta Yuri Gagarin, Valentina
Tereshkova y Neil Armstrong. Por otro lado la carrera espacial dejo de
ser una prioridad para estados-naci\'on y son compa\~nias privadas las
que hablan de turismo y posible colonizaci\'on de Marte en la
siguiente d\'ecada (ver nota de prensa en la Bibliograf\'ia
Complementaria). \\[-0.3cm]  

\noindent\textbf{\textsc{Semana 11.}} \textit{El espacio exterior en la
ciencia ficci\'on contempor\'anea). Paseo por la obra de Stanislaw Lem,
Isaac Asimov, Alejandro Jodorowsky y Carl Sagan y Philip K. Dick.}
Para esta semana vamos a leer varios fragmentos de  autores de ciencia
ficci\'on que han influenciado las  tem\'aticas y est\'eticas de
pel\'iculas y series de televisi\'on. Autores como Lem tienen un
mensaje filos\'ofico m\'as profundo, mientra que Jodorowsky ser\'a
discutido desde el punto de vista de sus comics (ver bibliograf\'ia
complementaria). \\[-0.3cm]  

\noindent\textbf{\textsc{Semana 12.}} \textit{De pel\'icula. Referencias en el
cine de ciencia ficci\'on a la exploraci\'on espacial y viaje
transdimensional}. Este es un tema extenso que podr\'ia cubrir varios
CBU. En esta semana nos vamos a centrar en ver y discutir fragmentos
de algunos cl\'asicos como \textit{2001: Space Oddisey}, \textit{Star
  Wars}, \textit{Solaris}, \textit{Alien} y 
\textit{E.T.}. Tambi\'en veremos y comentaremos fragmentos de
pel\'iculas m\'as recientes como \textit{Apollo 18}, \textit{Contact},
\textit{District 9}, \textit{Gravity} e
\textit{Interstellar}. \\[-0.3cm]    

\noindent\textbf{\textsc{Semana 13.}} \textit{Banda Sonora. Referencias de la
m\'usica popular a temas astron\'omicos. M\'usica espacial, La ciencia
detr\'as de sonorizaciones espacials.} Hay artistas que dicen venir de
otros planetas. El ejemplo m\'as famoso que merece ser mencionado es
Sun Ra, m\'usico de funk los 70 en Estados Unidos que dec\'ia venir de
Saturno (ver documental en la Bibliograf\'ia Complementaria). Esta
semana pasaremos la mayor parte del tiempo escuchando y comentando composiciones con
un tema espacial (i.e. Space Oddity, Space-Age Bachuelor Pad Music),
que merecen una discusi\'on sobre el contexto en el que fueron
creadas. Finalmente, hablaremos  sobre las sonorizaciones sobre
sonidos de misiones espaciales que son incluidos en obras musicales
(i.e. la colaboraci\'on de NASA con el Kronos Quartet, ver Bibliograf\'ia 
Complementaria).\\[-0.3cm]    

\noindent\textbf{\textsc{Semana 14.}} \textit{Space-Art. El punto de encuentro
entre la astronom\'ia y las ciencias del espacio con las artes
pl\'asticas.} Recorreremos la obra de artistas como Olafur Eliasson que se
integran conceptos de espacio, luz, astronom\'ia y cosmolog\'ia en sus
obras; hablaremos de festivales de arte como Kosmica que se dedican
a experimentar con mezclas de arte, ciencias, performance, video y
usos alternativos del espacio exterior (Ver notas de prensa en la
Bibliograf\'ia complementaria).  
\\[-0.3cm]  

\noindent\textbf{\textsc{Semana 15.}} \textit{El Fin. Historias de apocalipsis
astron\'omicos y su verdadera probabilidad de suceder.} Para terminar
hablaremos de diferentes tipos de finales astron\'omicos. El ejemplo
que guiar\'a la discusi\'on ser\'a el caso del 2012 con el apocalipsis
Maya (Ver nota de prensa en la Bibliograf\'ia Complementaria). Pero
hablaremos tambi\'en de extinciones masivas, la posibilidad de ser
devastados por un asteroide para terminar con una reflexi\'on sobre el
cambio clim\'atico a nivel global.\\[0.1cm] 

\stepcounter{mysection}
\noindent\textbf{\large \Roman{mysection} \quad Metodolog\'ia}\\[-0.2cm]

%Se describen las tecnicas y metodos para el desarrollo exitoso del curso.

\noindent\normalsize 
Cada semana los temas se presentar\'an por el instructor con
exposiciones orales y apoyo audiovisual (transparencias, audio,
video). En cada sesi\'on se presentar\'an dos aspectos del
conocimiento: el popular y el cient\'ifico. En cada tema se har\'a
\'enfasis en los aspectos que son apoyados o entran en contradicci\'on con el conocimiento cient\'ifico, as\'i como sobre
aquellas \'areas que se encuentran por fuera del alcance de la ciencia
contempor\'anea. El tono de las presentaciones ser\'a divulgativo,
presentando los conceptos importantes a trav\'es
de ejemplos y analog\'ias que no requieran un formalismo matem\'atico
complejo. Por su parte los estudiantes deben escribir tres ensayos
durante el curso. En tres sesiones del semestre se har\'an comentarios breves
(de media hora) sobre las entregas de los estudiantes. Por fuera de
clase se dar\'a una retroalimentaci\'on detallada por escrito a cada  ensayo.  
\\[0.1cm]

\stepcounter{mysection}
\noindent\textbf{\large \Roman{mysection} \quad Criterios de evaluaci\'on}\\[-0.2cm]

%Tener en cuenta los siguientes aspectos:
%	\item Porcentajes de cada evaluacion
%	\item Fechas importantes
%	\item Parametros de calificacion
%	\item Calificacion de asistencia y/o participacion en clase
%	\item Reclamos
%	\item Politica de aproximacion de notas

\noindent\normalsize 
El curso se calificar\'a con la entrega de tres ensayos de 2000 a 2500
palabras, cada uno de un valor del $33.3\%$ de la nota final.  Para
que los estudiantes tengan oportunidad de mejorar sus ensayos se
har\'an dos entregas. La primera servir\'a para dar una
retroalimentaci\'on inicial y tendr\'a una calificaci\'on de $10\%$
sobre la nota final. La segunda entrega ser\'a la definitiva y
tendr\'a  valor de $23.3\%$ sobre la nota final.  
 
El primer ensayo es un texto comparativo. Retoma un tema 
tratado en clase. El estudiante debe ampliar las referencias y la
discusi\'on hecha por el instructor.

El segundo ensayo es de car\'acter divulgativo. El estudiante retoma
un tema tratado en clase y lo convierte en un texto que podr\'ia
encontrarse en un medio de comunicaci\'on masivo.   

El tercer ensayo es de car\'acter argumentativo. El estudiante propone
su visi\'on sobre un tema nuevo que concuerda con el programa general
del curso. El ensayo debe tomar en cuenta una fuente de la cultura
popular (peri\'odicos, televisi\'on, radio, cine, c\'omics, museos) que
no haya sido explorada en profundidad durante el curso.   
\\[0.1cm]

\stepcounter{mysection}
\noindent\textbf{\large \Roman{mysection} \quad Bibliograf\'ia}\\[-0.2cm]

%Indicar los libros y la documentacion guia.


\noindent\normalsize Bibliograf\'ia principal:

\begin{itemize}
\item Barry Luokkala. \textit{Exploring Science Through Science
  Fiction}, 2014. (Biblioteca General - Recurso Electr\'onico 530.1 23).\\[-0.6cm]
\item Bryan A. Penprase. \textit{The power of stars: how celestial
  observations have shaped civilization}, 2011. (Biblioteca General -
  Recurso Electr\'onico 523.1) \\[-0.6cm]
\item Carl Sagan. \textit{Cosmos: una evoluci\'on c\'osmica de quince mil
  millones de a\~nos que ha transformado la materia en vida y en
  consciencia}, 1982. (Biblioteca General - 
  523.1 S131 Z258 )\\[-0.6cm]
\end{itemize}

\noindent\normalsize Bibliograf\'ia complementaria:


\begin{itemize}
\item Stanislaw Lem. \textit{Diarios de las estrellas : viajes y
  memorias}, 1978. (Biblioteca General - 891.8537 L25D Z251). \\[-0.6cm]
\item Roger D. Launius (Editor). \textit{Exploring the solar system:
  The history and science of planetary exploration},
  2013. (Biblioteca General - Recurso Electr\'onico 523.2 23 )\\[-0.6cm]
\item Carl Sagan. \textit{El mundo y sus demonios: la ciencia
  como una luz en la oscuridad}, Editorial Planeta,
  1997. (Biblioteca General - 001.94 S131 Z282) \\[-0.6cm]
\item Andrew M. Shaw. \textit{Astrochemistry: from astronomy to astrobiology}, 2011. (Biblioteca General -  523.02 S318 )\\[-0.6cm]
\item Charles Percy Snow. \textit{The two cultures},
  1993. (Biblioteca General - 301.2 S558T 1993)\\[-0.6cm]
\end{itemize}

\noindent
Notas de prensa y recursos de internet.

\begin{itemize}
\item How Different Cultures Understand Time. \\
\url{http://www.businessinsider.com/how-different-cultures-understand-time-2014-5} \\[-0.6cm]
\item Una parodia de periodistas escribiendo noticias
  cient\'ificas. This is a news website article about a scientific
  paper\\\url{http://www.theguardian.com/science/the-lay-scientist/2010/sep/24/1} \\[-0.6cm]
\item No, Mars Won't Be as Big as the Moon. Ever.\\
\url{http://www.slate.com/blogs/bad_astronomy/2013/08/22/mars_as_big_as_the_moon_no_just_no.html}.\\[-0.6cm]
\item SpaceX's Elon Musk to Reveal Mars Colonization Ideas This
  Year\\\url{http://www.space.com/28215-elon-musk-spacex-mars-colony-idea.html} \\[-0.6cm]
\item Video de Ulises I, el proyecto art\'istico para crear y lanzar
  el primer nanosat\'etlite hecho por
  ciudadanos.\\ \url{https://vimeo.com/99102519}.\\[-0.6cm]
\item KOSMICA 2014. Encuentro internacional de arte y ciencia que
  explora usos alternativos y culturales del
  espacio.\\ \url{http://www.centroculturadigital.mx/es/especial/kosmica-2014.html} \\[-0.6cm]
\item Jodorowsky's Dune\\\url{https://www.youtube.com/watch?v=jg4OCeSTL08}.\\[-0.6cm]
\item The Musical Sounds of Space. Kronos Quartet Performs Music Based
  on Distant  Signals\\
  \url{http://www.npr.org/templates/story/story.php?storyId=930399}\\[-0.6cm]
\item Sun Ra, Brother From Another Planet (BBC Documentary)\\
  \url{https://www.youtube.com/watch?v=AGTd3HR62fo} \\[-0.6cm]
\item Beyond 2012: Why the World Didn't End\\
  \url{http://www.nasa.gov/topics/earth/features/2012.html}. \\[-0.6cm]
\item Imagine Yourself as an Asteroid: Olafur Eliasson's Contact, at
  Fondation Louis
  Vuitton. \\\url{http://www.huffingtonpost.com/mutualart/imagine-yourself-as-an-as_b_6568710.html}. \\[-0.6cm] 
\end{itemize}


\end{document}
