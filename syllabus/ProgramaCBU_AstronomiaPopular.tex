\documentclass[letterpaper,10pt,onecolumn]{article}
\usepackage[spanish]{babel}
\usepackage[latin1]{inputenc}
\usepackage[pdftex]{color,graphicx}
\usepackage{hyperref}
\setlength{\oddsidemargin}{0cm}
\setlength{\textwidth}{490pt}
\setlength{\topmargin}{-40pt}
\addtolength{\hoffset}{-0.3cm}
\addtolength{\textheight}{4cm}

\begin{document}
\begin{center}

\includegraphics[width=490pt]{header.png}\\[0.5cm]

\textsc{\LARGE Astronom\'ia Popular}\\[0.1cm]

\large Jaime E. Forero Romero\\[0.5cm]

\end{center}

\large \noindent\textsc{Nombre del curso:} Astronom\'ia Popular %Aqui
                                %nombre del curso 
 
\noindent\textsc{C\'odigo del curso:} FISI-1916B. CBU B.

                                %codigo del curso 

\noindent\textsc{Unidad acad\'emica:} Departamento de F\'isica 

\noindent\textsc{Periodo acad\'emico:} 201620 %Aqui el periodo,
                                %p.ej. 201510 

\noindent\textsc{Horario:} Mi y Vi, 11:00 a 12:20 %Aqui el horario, p.ej. Ma y Ju, 10:00 a
                           %11:20 

\noindent\rule{\textwidth}{1pt}\\[-0.3cm]

\normalsize \noindent\textsc{Nombre profesor(a) principal:} Jaime
E. Forero-Romero %Aqui nombre del profesor principal 

\noindent\textsc{Correo electr\'onico:}
\href{mailto:je.forero@uniandes.edu.co}{\nolinkurl{je.forero@uniandes.edu.co}}
%Cambie address por su direccion de correo uniandes 

\noindent\textsc{Horario y lugar de atenci\'on:} Mi 14:00 a 17:00, Oficina Ip208 %Aqui su
                                %horario y lugar de atencion,
                                %p.ej. Vi, 15:00 a 17:00, Oficina
                                %Ip102 
\\[-0.1cm]


\noindent\rule{\textwidth}{1pt}\\[-0.1cm]

\newcounter{mysection}
\addtocounter{mysection}{1}

\noindent\textbf{\large \Roman{mysection} \quad
  Introducci\'on}\\[-0.2cm] 

%Este espacio es para hacer una introduccion al curso, evidenciando la propuesta metodologica. Debe ser clara y precisa.

\noindent\normalsize 
La astronom\'ia est\'a presente en la informaci\'on que
recibimos a diario en el Siglo XXI. Posible vida en otros planetas, el
origen del  Universo, apocalipsis astron\'omicos, ciencia ficci\'on,
celebraciones rituales en el calendario y la exploraci\'on espacial
son algunos de los temas que parecen familiares a la mayor\'ia pero que
en realidad no los conocemos en toda la profundidad que merecen. El
objetivo principal del curso es tratar aspectos de la cultura popular
relacionados con la astronom\'ia y presentarlos a trav\'es del
conocimiento que se ha constru\'ido en cosmolog\'ia, astronom\'ia
gal\'actica, astrof\'isica planetaria, astrof\'isica estelar,
astronom\'ia de posici\'on, coheter\'ia, ingenier\'ia aeroespacial y
astrobiolog\'ia. \\[0.1cm]  

\stepcounter{mysection}
\noindent\textbf{\large \Roman{mysection} \quad Objetivos}\\[-0.2cm]

%En este espacio se debe precisar el ente visor del curso y el proposito ideal al finalizar el curso.

\noindent\normalsize Los objetivos principales del curso son:

\begin{itemize}
\item Reconocer el lugar del conocimiento astron\'omico en diferentes
  aspectos de la cultura popular. \\[-0.6cm] 

\item Examinar los v\'inculos hist\'oricos de la astronom\'ia con
  otras \'areas del conocimiento.  \\[-0.6cm] 

\item Reconocer diferentes maneras de construir y validar conocimiento
  en las ciencias naturales, en particular en astronom\'ia.\\[-0.2cm]
\end{itemize}

\stepcounter{mysection}
\noindent\textbf{\large \Roman{mysection} \quad Competencias a
  desarrollar}\\[-0.2cm] 

%En este espacio se describen las habilidades que el estudiante
%desarrollara en el transcurso del curso. 

\noindent\normalsize Al finalizar el curso, se espera que el
estudiante est\'e en capacidad de: 

\begin{itemize}
\item Analizar contenidos de diferentes fuentes en la cultura popular
  para encontrar su relaci\'on con el conocimiento astron\'omico.\\[-0.6cm]
\item Leer textos no especializados con contenido astron\'omico para
  reconocer el posible uso correcto o incorrecto de conceptos cient\'ificos.
\\[-0.6cm]
\item Escribir un texto sobre alg\'un aspecto de la cultura popular
  influenciado por la astronom\'ia para evidenciar la relaci\'on del
  conocimiento cient\'ifico con las creencias/valores del p\'ublico no
  especializado.\\[-0.2cm]  
\end{itemize}

\newpage
\stepcounter{mysection}
\noindent\textbf{\large \Roman{mysection} \quad Contenido por
  semanas}\\[-0.2cm] 


%Se expone de forma ordenada toda la tematica a tratar del curso. Debe
%planearse para 15 semanas. 

\noindent\normalsize \textbf{\textsc{Semana 1.}} \textit{El Comienzo.} Las
diferentes historias de origen en los \'ultimos dos mil a\~nos
en diferentes civilizaciones de Asia, Africa y Am\'erica. La teor\'ia
del Big Bang. Las historias de or\'igen son parte integral de una
civilizaci\'on. Nuestras nociones impl\'icitas sobre la creaci\'on
nos vinculan. \\[-0.3cm]   


\noindent\textbf{\textsc{Semana 2.}} \textit{Perspectivas Cu\'anticas}
Presentaci\'on de fen�menos b\'asicos sobre la descripci\'on de la
materia a trav\'es de la mec\'anica cu\'antica. Influencia de estos
fen\'omenos en los or\'igenes de nuestro Universo. Contraste de estos
conceptos cient\'ificos de materia y energ�a con visiones del
misticismo oriental. 

\noindent\textbf{\textsc{Semana 3.}} \textit{El Cuerpo y el Universo}.
Relaciones f\'isicas entre nuestro cuerpo y el Universo. �C\'omo afecta
nuestro nuevo conocimiento del cosmos nuestra vida diaria? Relaci\'on de
este conocimientos con hechos m\'iticos religiosos. ?`Necesitamos crear
una nueva mitolog\'ia a partir de estos nuevos conocimientos? 


\noindent\normalsize \textbf{\textsc{Semana 4.}} \textit{Arte,
  astronom�a y ritual.} Puntos de encuentro del arte, la astronom�a y
el ritual religioso en los �ltimos tres mil a�os de historia
global. Haremos �nfasis en las representaciones de religiones
solares, como el cristianismo. 

\noindent\textbf{\textsc{Semana 5.}} \textit{Espacio y tiempo.} Modelos de
espacio y tiempo desde Grecia antigua hasta Einstein. Los conceptos de
espacio y de tiempo son partes integrales de la cultura.
Mostraremos c\'omo la forma en la que se mide y la forma en la que se
crea una estructura mental de estos dos conceptos define la
posici\'on ante la vida de sociedades completas. \\[-0.3cm]  

\noindent\textbf{\textsc{Semana 6.}} \textit{Ritmos de la vida diaria.} Ciclos
lunares y solares que dictan ritmos biol\'ogicos y
dan estructura a sociedades. 
Revisaremos las diferentes posibles definiciones y mediciones de
estos ciclos y su influencia sobre los ritmos biol\'ogicos
(i.e. producci\'on de hormonas estimulados por la luz
solar). 
Igualmente revisaremos los ciclos humanos que tienen una periodicidad
lunar en agricultura y en ritmos biol\'ogicos (i.e. el ciclo menstrual
de las mujeres). \\ [-0.3cm]    


\noindent\textbf{\textsc{Semana 7.}} \textit{Signos.} Lo que la ciencia
contempor\'anea tiene que decir sobre la astrolog\'ia. 
�`Qu� realidades se est\'an manifestando a trav\'es del inter\'es de la gente
en la astrolog\'ia? Repaso sobre los eventos astron\'omicos tangibles
que pueden influir sobre la poblaci\'on humana (i.e ciclos de
actividad solar y meteoritos).\\[-0.3cm]     

\noindent\textbf{\textsc{Semana 8.}}\textit{Poder.} Desde
los emperadores hasta la carrera espacial. El espacio ha jugado un rol
importante en la pol\'itica terrestre. \\[-0.3cm]  

\noindent\textbf{\textsc{Semana 9.}} \textit{El Universo visto por los
periodistas.} Los errores y aciertos de los periodistas en
cubrimientos de noticias sobre astronom\'ia. 
Revisaremos la manera en la que fen\'omenos relacionados con el
espacio son representados en los medios masivos de
comunicaci\'on). \\[-0.3cm]    


\noindent\textbf{\textsc{Semana 10.}} \textit{OVNIS}. ?`Existen OVNIS y
marcianos?. Astrobiolog\'ia b\'asica y la b\'usqueda cient\'ifica de
vida  extraterrestre y mundos habitables.  
\\[-0.3cm]   

\noindent\textbf{\textsc{Semana 11.}} \textit{El espacio exterior en la
ciencia ficci\'on contempor\'anea).} Paseo por la obra de Stanislaw Lem,
Isaac Asimov, Alejandro Jodorowsky y Carl Sagan y Philip K. Dick. 
\\[-0.3cm]  

\noindent\textbf{\textsc{Semana 12.}} \textit{De pel\'icula.} Referencias en el
cine de ciencia ficci\'on a la exploraci\'on espacial y viaje
transdimensional. 
Haremos un resumen hist\'orico de diferentes maneras en la que el cine
a utilizado e interpretado los fen\'omenos espaciales.
\\[-0.3cm]    

\noindent\textbf{\textsc{Semana 13.}} \textit{Banda Sonora}. Referencias de la
m\'usica popular a temas astron\'omicos. M\'usica espacial, La ciencia
detr\'as de sonorizaciones espaciales y  artistas que dicen venir de
otros planetas. \\[-0.3cm]    


\noindent\textbf{\textsc{Semana 14.}} \textit{Arte, Astronom\'ia y
  Ritual. (II)} El punto de encuentro entre la astronom\'ia y las
ciencias del espacio con las artes pl\'asticas contempor\'aneas. 
Recorreremos la obra de artistas y colectivos que toman como fuente
central a la astronom\'ia.
\\[-0.3cm]  

\noindent\textbf{\textsc{Semana 15.}} \textit{El Fin.} 
Significado de diferentes historias de apocalipsis y fin del
mundo. Posibles visiones de El Fin desde la astronom�a. `�Qu\'e valor
tiene una narrativa sobre El Fin para nuestra vida diaria? \\[0.1cm] 

\stepcounter{mysection}
\noindent\textbf{\large \Roman{mysection} \quad Metodolog\'ia}\\[-0.2cm]

%Se describen las tecnicas y metodos para el desarrollo exitoso del curso.

\noindent\normalsize 
Cada semana los temas se presentar\'an por el instructor con
exposiciones orales y apoyo audiovisual (transparencias, audio,
video). En cada sesi\'on se presentar\'an dos aspectos del
conocimiento: el popular y el cient\'ifico. En cada tema se har\'a
\'enfasis en los aspectos que son apoyados o entran en contradicci\'on con el conocimiento cient\'ifico, as\'i como sobre
aquellas \'areas que se encuentran por fuera del alcance de la ciencia
contempor\'anea. El tono de las presentaciones ser\'a divulgativo,
presentando los conceptos importantes a trav\'es
de ejemplos y analog\'ias que no requieran un formalismo matem\'atico
complejo. Por su parte los estudiantes deben escribir tres ensayos
durante el curso. En tres sesiones del semestre se har\'an comentarios breves
(de media hora) sobre las entregas de los estudiantes. Por fuera de
clase se dar\'a una retroalimentaci\'on detallada por escrito a cada  ensayo.  
\\[0.1cm]

\stepcounter{mysection}
\noindent\textbf{\large \Roman{mysection} \quad Criterios de evaluaci\'on}\\[-0.2cm]

%Tener en cuenta los siguientes aspectos:
%	\item Porcentajes de cada evaluacion
%	\item Fechas importantes
%	\item Parametros de calificacion
%	\item Calificacion de asistencia y/o participacion en clase
%	\item Reclamos
%	\item Politica de aproximacion de notas

\noindent\normalsize 
El curso se calificar\'a con la entrega de tres textos argumentativos
de al menos 500 palabras. 
Estos textos se redactar\'an en clase. 
En el texto cada estudiante debe defender una posici\'on con respecto
a una pregunta o un texto planteado al comienzo del ejercicio de
escritura. 
Cada uno de los textos tiene un valor del $33.3 \%$ de la nota final. 

En cada texto se calificar\'a la claridad y coherencia de:
\begin{itemize}
\item el texto en general  ($40\%$ de la nota),
\item la idea principal ($50\%$ de la nota),
\item las ideas secundarias que apoyan la idea principal  ($50\%$ de
  la nota).  
\end{itemize}
Cada error de gram\'atica, ortograf\'ia o palabra por debajo de las
500 palabras solicitadas resta a la nota $0.01\%$. 
No hay bonos por palabras adicionales. 
\\[0.1cm]

Las fechas de redacci\'on de los textos son:
\begin{itemize}
\item Septiembre 2, 2016.
\item Octubre 7, 2016.
\item Noviembre 4, 2016.
\end{itemize}

\stepcounter{mysection}
\noindent\textbf{\large \Roman{mysection} \quad Bibliograf\'ia}\\[-0.2cm]

%Indicar los libros y la documentacion guia.


\noindent\normalsize Bibliograf\'ia principal:

\begin{itemize}
\item Barry Luokkala. \textit{Exploring Science Through Science
  Fiction}, 2014. (Biblioteca General - Recurso Electr\'onico 530.1 23).\\[-0.6cm]
\item Bryan A. Penprase. \textit{The power of stars: how celestial
  observations have shaped civilization}, 2011. (Biblioteca General -
  Recurso Electr\'onico 523.1) \\[-0.6cm]
\item Carl Sagan. \textit{Cosmos: una evoluci\'on c\'osmica de quince mil
  millones de a\~nos que ha transformado la materia en vida y en
  consciencia}, 1982. (Biblioteca General - 
  523.1 S131 Z258 )\\[-0.6cm]
\end{itemize}

\noindent\normalsize Bibliograf\'ia complementaria:


\begin{itemize}
\item Stanislaw Lem. \textit{Diarios de las estrellas : viajes y
  memorias}, 1978. (Biblioteca General - 891.8537 L25D Z251). \\[-0.6cm]
\item Roger D. Launius (Editor). \textit{Exploring the solar system:
  The history and science of planetary exploration},
  2013. (Biblioteca General - Recurso Electr\'onico 523.2 23 )\\[-0.6cm]
\item Carl Sagan. \textit{El mundo y sus demonios: la ciencia
  como una luz en la oscuridad}, Editorial Planeta,
  1997. (Biblioteca General - 001.94 S131 Z282) \\[-0.6cm]
\item Andrew M. Shaw. \textit{Astrochemistry: from astronomy to astrobiology}, 2011. (Biblioteca General -  523.02 S318 )\\[-0.6cm]
\item Charles Percy Snow. \textit{The two cultures},
  1993. (Biblioteca General - 301.2 S558T 1993)\\[-0.6cm]
\end{itemize}


\end{document}
